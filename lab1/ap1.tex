%%%%%%%%%%%%%%%%%%%%%%%%%%%%%%%%%%%%%%%%%%%%%%%%%%%%%%%%%%%%%%%%%%%
% 
% $Id: ap1.tex,v 1.1.1.1 2002/01/02 19:36:28 phil Exp $
%
% $Log: ap1.tex,v $
% Revision 1.1.1.1  2002/01/02 19:36:28  phil
% initial import into CVS
%
% Revision 1.4  1996/08/13 00:56:10  cguo
% glossary a problem
%
% Revision 1.3  1996/08/12 23:25:32  cguo
% ready for phil
%
% Revision 1.2  1995/08/14 22:24:47  stockie
% - change labels to depend on lab #
%
% Revision 1.1  1995/07/18  21:39:32  stockie
% Initial revision
%
%
%%%%%%%%%%%%%%%%%%%%%%%%%%%%%%%%%%%%%%%%%%%%%%%%%%%%%%%%%%%%%%%%%%%

\section{Mathematical Notes}
\label{lab1:ap:mathnote}

\subsection{Solution to the Heat Conduction Equation}
\label{lab1:ap:conduction}

In Example~\ref{lab1:exm:conduction}, we had the equation
\[
  \frac{dT}{dt} = -\lambda (T-T_a),
\]
subject to the initial condition $T(0)$.  This equation can be solved
by \emph{separation of variables}, whereby all expressions involving
the independent variable $t$ are moved to the right hand side, and all
those involving the dependent variable $T$ are moved to the left
\[
  \frac{dT}{T-T_a} = -\lambda dt.
\]
The resulting expression is integrated from time $0$ to $t$
\[
  \int_{T(0)}^{T(t)} \frac{dS}{S-T_a} = -\int_0^t\lambda ds,
\]
(where $s$ and $S$ are dummy variables of integration), which then
leads to the relationship
\[
  \ln \left( T(t)-T_a)-\ln(T(0)-T_a \right) = -\lambda t,
\]
or, after exponentiating both sides and rearranging,
\[
  T(t) = T_a + (T(0)-T_a)e^{-\lambda t},
\]
which is exactly equation~\eqref{lab1:eq:conduction-soln}.

%%% Local Variables: 
%%% mode: latex
%%% TeX-master: "lab1"
%%% End: 
