%%%%%%%%%%%%%%%%%%%%%%%%%%%%%%%%%%%%%%%%%%%%%%%%%%%%%%%%%%%%%%%%%%%
% 
% $Id: ap2.tex,v 1.1.1.1 2002/01/02 19:36:28 phil Exp $
%
% $Log: ap2.tex,v $
% Revision 1.1.1.1  2002/01/02 19:36:28  phil
% initial import into CVS
%
% Revision 1.2  1996/08/12 23:25:45  cguo
% ready for phil
%
% Revision 1.1  1995/08/14 22:25:05  stockie
% Initial revision
%
%
%%%%%%%%%%%%%%%%%%%%%%%%%%%%%%%%%%%%%%%%%%%%%%%%%%%%%%%%%%%%%%%%%%%

\section{Technical Notes}
\label{lab1:ap:technote}

\subsection{Conduction Demo}
\label{lab1:tech:conduction}

The conduction demo is generated using two CGI scripts:
\begin{enumerate}
\item \htmladdnormallinkfoot{{\tt
    conduction.cgi}}{\cgibinURL{conduction.cgi.source}},
  which does the following:
  \begin{itemize}
  \item sets up forms to take values of the problem parameters from the
    student.  
  \item accumulates different values of the initial temperature, so that
    several plots can be displayed at once for fixed values of the
    ambient temperature, final time and $\lambda$.  
  \item builds a list of command line parameters that will be passed to
    the plotting script.
    \item invokes the plotting script, {\tt conduction2.cgi}, by
      reading it in as an image; the output from this script will then
      be read inline as a GIF image.
    \end{itemize}
\item \htmladdnormallinkfoot{{\tt
    conduction2.cgi}}{\cgibinURL{conduction2.cgi.source}}, which does
  the following:
  \begin{itemize}
  \item parses the parameters from the first script.
  \item opens a stream to Gnuplot and writes plotting commands based
    on the parameter values.
  \item writes the GIF file from Gnuplot to standard output, which
    is the image displayed by the first script.
  \end{itemize}
\end{enumerate}

\subsection{Interpolation Demo}
\label{lab1:tech:discrete}

This demo uses the following three CGI scripts:
\begin{enumerate}
\item \htmladdnormallinkfoot{{\tt
  discrete.cgi}}{\cgibinURL{discrete.cgi.source}}, which presents a
form for the student to enter the function, interpolation scheme and
number of points to use in the interpolation.  This information is set
up in an argument string that is passed to one of the following two
scripts, depending on which function is chosen.
\item \htmladdnormallinkfoot{{\tt
  fplot.cgi}}{\cgibinURL{fplot.cgi.source}}, which uses Gnuplot to plot
the interpolated function $f(x)$.  The script generates a list of data, and
uses Gnuplot's ``lines'' data style to perform the linear
interpolation. 
\item \htmladdnormallinkfoot{{\tt
  gplot.cgi}}{\cgibinURL{gplot.cgi.source}}, which is the same as {\tt
  fplot.cgi}, except that it uses the ``csplines'' data style in Gnuplot
to interpolate $g(x)$ using cubic splines.
\end{enumerate}

\subsection{Derivative Demo}
\label{lab1:tech:deriv}

This demo uses the single CGI script 
\htmladdnormallinkfoot{{\tt
  deriv.cgi}}{\cgibinURL{deriv.cgi.source}}, which keeps a 
count of the number of times the student has clicked on the ``GO'' 
button, which determines which frame of the derivative ``movie'' to
display.  
The GIF image files for each frame have already been computed.

%%% Local Variables: 
%%% mode: latex
%%% TeX-master: "lab1"
%%% End: 
