%%%%%%%%%%%%%%%%%%%%%%%%%%%%%%%%%%%%%%%%%%%%%%%%%%%%%%%%%%%%%%%%%%%
% 
% $Id: sec1.tex,v 1.1.1.1 2002/01/02 19:36:28 phil Exp $
%
% $Log: sec1.tex,v $
% Revision 1.1.1.1  2002/01/02 19:36:28  phil
% initial import into CVS
%
% Revision 1.6  1996/08/13 00:56:11  cguo
% glossary a problem
%
% Revision 1.5  1996/08/06 21:38:53  cguo
% *** empty log message ***
%
% Revision 1.4  1996/08/02 22:17:41  cguo
% more detailed objectives
%
% Revision 1.3  1996/04/29 19:02:38  stockie
% ready for carmen
%
% Revision 1.2  1995/08/14  20:29:19  stockie
% - add link to lab 2
%
% Revision 1.1  1995/07/18  21:38:01  stockie
% Initial revision
%
%
%%%%%%%%%%%%%%%%%%%%%%%%%%%%%%%%%%%%%%%%%%%%%%%%%%%%%%%%%%%%%%%%%%%

\section{Objectives}

The examples and exercises in this lab are meant to illustrate the
limitations of analytical solution techniques, using
several differential equation models for simple physical systems.
This is the prime motivation for the use of numerical methods.

After completing this lab, you will understand the process of
\emph{discretizing} a continuous problem, and be able to derive a simple
finite difference approximation for an ordinary or partial
differential equation.
The examples will also introduce the concepts of \emph{accuracy} and
\emph{stability}, which will be discussed further in the 
\htmladdnormallink{following lab}{\LabtwoURL}.

Specifically you will be able  to:
\begin  {itemize}
\item Define the term or identify: Ordinary Differential Equation,
Partial Differential Equation, Linear equation, Non-linear equation,
Initial value problem,Boundary value problem, Open Domain, and Closed
Domain.

\item Define the term, identify or perform: Forward difference discretization,
Backward difference discretization, and Centre difference discretization.

\item Define the term: Interpolation, Convergence, and Instability.

\item Define the term or perform: Linear interpolation.

\end {itemize}

%%% Local Variables: 
%%% mode: latex
%%% TeX-master: "lab1"
%%% End: 
