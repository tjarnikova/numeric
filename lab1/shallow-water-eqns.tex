\begin{example}
  Many flows in the atmosphere and ocean are governed by the
  Navier-Stokes equations, which are a complicated, non-linear system
  of PDE's, relating the evolution of pressure and velocity for a
  viscous fluid.
  Let us consider a simplified model of fluid motion, in
  which the flow is in the $x$-direction only, and pressure effects
  are ignored.  For this situation, the Navier-Stokes equations reduce
  to a single, non-linear partial differential equation in the
  velocity only:
  \[
    \underbrace{u_t}_{\mbox{time-dependence}} +
    \underbrace{uu_x}_{\mbox{advection}} 
    = \underbrace{\nu u_{xx}}_{\mbox{diffusion}}, 
  \]
  where $\nu$ is the {\em kinematic viscosity}.
  This equation is commonly known as {\em Burger's equation}. 

  \begin{mathnote}
    \hyperref{Details on the derivation from the Navier-Stokes
      equations \dots}{See Appendix }{ for an 
      an overview of the Navier-Stokes equations and the
      derivation of Burger's equation.}{lab1:ap:burgers} 
  \end{mathnote}

  In order that the problem have a unique solution, we must also
  impose the initial values
  \[
    u(x,0) = u_0(x).
  \]
  If we look for a solution without imposing any boundaries, and take
  $\nu$ to be constant, then the exact solution can be found as 
  \[
    u(x,t) = -2\nu \log \left[ \frac{1}{2 \sqrt{\pi \nu t}}
      \int_{-\infty}^\infty e^{-u_0(y)/2\nu} e^{-(x-y)^2/4\nu t}dy\right]. 
  \]
  If you thought that the solution to the heat equation problem in the
  previous example was impractical, then this solution is even worse!

  \begin{mathnote}
    \hyperref{Details of the derivation \dots}{See Appendix }{ for an  
      a derivation of the solution.}{lab1:ap:burgers-soln} 
  \end{mathnote}
\end{example}

