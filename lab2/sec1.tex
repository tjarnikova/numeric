%%%%%%%%%%%%%%%%%%%%%%%%%%%%%%%%%%%%%%%%%%%%%%%%%%%%%%%%%%%%%%%%%%%
% 
% $Id: sec1.tex,v 1.1.1.1 2002/01/02 19:36:48 phil Exp $
%
% $Log: sec1.tex,v $
% Revision 1.1.1.1  2002/01/02 19:36:48  phil
% initial import into CVS
%
% Revision 1.3  1997/08/28 16:40:03  cguo
% *** empty log message ***
%
% Revision 1.2  1996/04/29 19:05:49  stockie
% ready for carmen
%
% Revision 1.1  1995/08/29  21:06:41  stockie
% Initial revision
%
% Revision 1.1  1995/08/29  21:06:41  stockie
% Initial revision
%
% Revision 1.2  1995/06/29  21:26:21  stockie
% *** empty log message ***
%
%
%%%%%%%%%%%%%%%%%%%%%%%%%%%%%%%%%%%%%%%%%%%%%%%%%%%%%%%%%%%%%%%%%%%

\section{Objectives}

In \htmladdnormallink{Lab~\#1}{\LaboneURL}, you were introduced to the
concept of discretization, and saw that there were
many different ways to approximate a given problem.
This Lab will delve further into the concepts of accuracy and
stability of numerical schemes, in order that we can compare the many
possible discretizations.

At the end of this Lab, you will have seen where the error in a
numerical scheme comes from, and how to quantify the error in terms of
\emph{ order}.  
The stability of several examples will be demonstrated, so that you
can recognize when a scheme is unstable, and how one might go about
modifying the scheme to eliminate the instability.

Specifically you will be able to:
\begin {itemize}

\item Define the term and identify: Implicit numerical scheme and
Explicit numerical scheme.

\item Define the term, identify, or write down for a given equation:
Backward Euler method and Forward Euler method.

\item Explain the difference in terminology between:
Forward difference discretization and Forward Euler method.

\item Define: truncation error, local truncation error, global truncation error,
and stiff equation.

\item Explain:  a predictor-corrector method.

\item Identify from a plot: an unstable numerical solution.

\item Be able to: find the order of a scheme, using the test equation
find the stability of a scheme, find the local truncation error from
a graph of the exact solution and the numerical solution.

\end {itemize}

%%% Local Variables: 
%%% mode: latex
%%% TeX-master: "lab2"
%%% End: 
