%%%%%%%%%%%%%%%%%%%%%%%%%%%%%%%%%%%%%%%%%%%%%%%%%%%%%%%%%%%%%%%%%%%
% 
% $Id: sec3.tex,v 1.1.1.1 2002/01/02 19:36:48 phil Exp $
%
% $Log: sec3.tex,v $
% Revision 1.1.1.1  2002/01/02 19:36:48  phil
% initial import into CVS
%
% Revision 1.4  1997/08/28 16:40:05  cguo
% *** empty log message ***
%
% Revision 1.3  1996/04/29 19:06:57  stockie
% ready for carmen
%
% Revision 1.2  1995/08/29  21:22:36  stockie
% *** empty log message ***
%
% Revision 1.2  1995/06/29  21:26:21  stockie
% *** empty log message ***
%
%
%%%%%%%%%%%%%%%%%%%%%%%%%%%%%%%%%%%%%%%%%%%%%%%%%%%%%%%%%%%%%%%%%%%

\section{Introduction}
\label{lab2:sec:intro}

Remember from Lab~\#1~\externalref{lab1:sec:diff-first-deriv} that you
were introduced to three approximations to the first derivative of a
function, $T^\prime(t)$.  If the independent variable, $t$, is
discretized at a sequence of N points, $t_i=t_0+i \Delta t$, where $i
= 0,1,\ldots, N$ and $\Delta t= 1/N$, then we can write the three
approximations as follows:
\begin{description}
\item[\textbf{ Forward difference formula:}] 
  \begin{equation}
    T^\prime(t_i) \approx \frac{T_{i+1}-T_i}{\Delta t},
    \label{lab2:eq:forward-diff}
  \end{equation}
\item[\textbf{ Backward difference formula:}]
  \begin{equation}
    T^\prime(t_i) \approx \frac{T_{i}-T_{i-1}}{\Delta t},
    \label{lab2:eq:backward-diff}
  \end{equation}
\item[\textbf{ Centered difference formula:}]
  \begin{equation}
    T^\prime(t_i) \approx \frac{T_{i+1}-T_{i-1}}{2 \Delta t}.
    \label{lab2:eq:centered-diff}
  \end{equation}
\end{description}
In fact, there are many other possible methods to approximate the
derivative (some of which we will see later in this Lab).
With this large choice we have in the choice of approximation scheme,
it is not at all clear at this point which, if any, of the schemes 
is the ``best''.
It is the purpose of this Lab to present you with some basic tools
that will help you to decide on an appropriate discretization for a
given problem.
There is no generic ``best'' method, and the choice of discretization
will always depend on the problem that is being dealt with.

In an example from Lab~\#1,\externalref{lab1:exm:saturation} the
forward difference formula was used to compute solutions to the
saturation development equation,\externalref{lab1:eq:saturation} and
you saw two important results:
\begin{itemize}
\item reducing the grid spacing, $\Delta t$, seemed to improve the
  accuracy of the approximate solution; and  
\item if $\Delta t$ was taken too large (that is, the grid was not
  fine enough), then the approximate solution exhibited non-physical
  oscillations, or a \emph{  numerical instability}.
\end{itemize}
There are several questions that arise from this example:
\begin{enumerate}
\item Is it always true that reducing $\Delta t$ will improve the
  discrete solution?
\item Is it possible to improve the accuracy by using another
  approximation scheme (such as one based on the backward or centered
  difference formulas)? 
\item Are these numerical instabilities something that always appear
  when the grid spacing is too large?  
\item By using another difference formula for the first
  derivative, is it possible to improve the stability of the
  approximate solution, or to eliminate the stability altogether?
\end{enumerate}
The first two questions, related to \emph{ accuracy}, will be dealt
with in Section~\ref{lab2:sec:accuracy-main}, and the last two will have to
wait until Section~\ref{lab2:sec:stability} when \emph{ stability} is
discussed.


%%% Local Variables: 
%%% mode: latex
%%% TeX-master: "lab2"
%%% End: 
