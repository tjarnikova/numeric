%%%%%%%%%%%%%%%%%%%%%%%%%%%%%%%%%%%%%%%%%%%%%%%%%%%%%%%%%%%%%%%%%%%
% 
% $Id: sec5.tex,v 1.1.1.1 2002/01/02 19:36:48 phil Exp $
%
% $Log: sec5.tex,v $
% Revision 1.1.1.1  2002/01/02 19:36:48  phil
% initial import into CVS
%
% Revision 1.3  1997/08/28 16:40:07  cguo
% *** empty log message ***
%
% Revision 1.2  1996/04/29 19:06:27  stockie
% ready for carmen
%
% Revision 1.1  1995/08/29  21:06:41  stockie
% Initial revision
%
% Revision 1.2  1995/06/29  21:26:21  stockie
% *** empty log message ***
%
%
%%%%%%%%%%%%%%%%%%%%%%%%%%%%%%%%%%%%%%%%%%%%%%%%%%%%%%%%%%%%%%%%%%%

\section{Stability of Difference Approximations}
\label{lab2:sec:stability}

The easiest way to introduce the concept of stability is for you to
see it yourself.

\begin{problem}
  \label{lab2:prob:stability}
  This example is a slight modification of
  Problem~\ref{lab2:prob:accuracy} from the previous section on
  accuracy.  We will add one scheme (backward euler) and drop the
  4th order Runge-Kutta, and change the focus from error to
  stability.  The value of $\lambda$ is assumed a constant, so that the
  backward Euler scheme results in an explicit method, and we'll also
  compute a bit further in time, so that any instability manifests
  itself more clearly.  Run the stability2.m or stability2.py
  scripts with $\lambda= -8\ s^{-1}$, with $\Delta t$ values that
  just straddle the stability condition for the forward euler
  scheme ($\Delta t < \frac{-2}{\lambda}$, derived below).  Hand in plots with comments that show that
  1) the stability condition does in fact predict the onset of the instablity
  in the euler scheme, and 2) the backward euler and leapfrog are
  either stable or unstable for the same $\Delta t$ values. (you
  should run out to longer than tend=10 seconds to see if there
  is a delayed instability.)

  \end{problem}

The heat conduction problem, as you saw in
Lab~\#1,\externalref{lab1:demo:conduction} has solutions that are
stable when $\lambda<0$.  
It is clear from Problem~\ref{lab2:prob:stability} above that some
higher order schemes 
(namely, the leap-frog scheme) introduce a spurious oscillation not
present in the continuous solution.
This is called a \emph{ computational} or \emph{ numerical instability},
because it is an artifact of the discretization process only.
This instability is not a characteristic of the heat conduction
problem alone, but is present in other problems where such schemes are
used.  Furthermore, as we will see below, even a scheme such as
forward Euler can be unstable for certain problems and choices of the
time step.

There is a way to determine the stability properties of a scheme, and
that is to apply the scheme to the \emph{ test equation}
\begin{equation}
\frac{dz}{dt} = \lambda z, \label{lab2:eq:test-equation}
\end{equation}
where $\lambda$ is a complex constant.
\begin{note}
The reason for using this
equation may not seem very clear.  But if you think in terms of
$\lambda z$ as being the linearization of some more complex right hand
side, then the solution to \eqref{lab2:eq:test-equation} is
$z=e^{\lambda t}$, and so $z$ 
represents, in some sense, a Fourier 
mode of the solution to the linearized ODE problem.  
We expect that the behaviour of the simpler, linearized problem should
mimic that of the original problem.
\end{note}  
Applying the forward Euler scheme to this test equation, results  in
the following difference formula
\[ z_{i+1} = z_i+(\lambda \Delta t)z_i \]
which is a formula that we can apply iteratively to $z_i$ to obtain
\begin{eqnarray*}
z_{i+1} &=& (1+\lambda \Delta t)z_{i} \\
        &=& (1+\lambda \Delta t)^2 z_{i-1} \\
        &=& \cdots \\
        &=& (1+\lambda \Delta t)^{i+1} z_{0}.
\end{eqnarray*}
The value of $z_0$ is fixed by the initial conditions, and so this
difference equation for $z_{i+1}$ will ``blow up'' as $i$ gets bigger,
if the factor in front of $z_0$ is greater than 1 in magnitude -- this
is a sign of instability.
Hence, this analysis has led us to the conclusion that if 
\[
|1+\lambda\Delta t| < 1,
\]
then the forward Euler method is stable.
For \emph{ real} values of $\lambda<0$, this inequality can be shown to be
equivalent to the \emph{ stability condition}
\[
\Delta t < \frac{-2}{\lambda},
\]
which is a restriction on how large the time step can be so that the
numerical solution is stable.

\begin{problem}
  \label{lab2:prob:test-backward-euler}
  Perform a similar analysis for the backward Euler formula, and show
  that it is \emph{ always stable} when $\lambda$ is real and
  negative.   
\end{problem}

\begin{example}
  \label{lab2:exm:test-leap-frog}
  \emph{ Now, what about the leap frog scheme?}

  Applying the test equation \eqref{lab2:eq:test-equation} to the leap
  frog scheme results in the difference equation
  \[
  z_{i+1} = z_{i-1} + 2 \lambda \Delta t z_i.
  \]
  Difference formulas such as this one are typically solved by looking
  for a solution of the form $z_i = w^i$ which, when substituted into
  this equation, yields
  \[ w^2 - 2\lambda\Delta t w - 1 = 0,\]
  a quadratic equation with solution
  \[ 
    w = \lambda \Delta t \left[ 1 \pm \sqrt{1+\frac{1}{(\lambda
        \Delta t)^2}} \right].
  \]
  The solution to the original difference equation, $z_i=w^i$ is
  stable only if all solutions to this quadratic satisfy $|w|<1$,
  since otherwise, $z_i$ will blow up as $i$ gets large.
 
  The mathematical details are not important here -- what is important
  is that there are two (possibly complex) roots to the quadratic
  equation for $w$,
  and one is \emph{ always} greater than 1 in magnitude \emph{ unless}
  $\lambda$ is pure imaginary (\ie~has real part equal to zero), \emph{
    and} $|\lambda \Delta t|<1$. 
  For the heat conduction equation in
  Problem~\ref{lab2:prob:stability} (which is already of the same form as
  the test equation \eqref{lab2:eq:test-equation}), $\lambda$ is
  clearly not imaginary, which explains the presence of the
  instability for the leap-frog scheme.

  Nevertheless, the leap frog scheme is still useful for
  computations.  In fact, it is often used in geophysical
  applications, as you will see later on when discretizing 
  \htmladdnormallinkfoot{the shallow water wave equations in
    Lab~\#7}{\LabsevenURL}, and
  \htmladdnormallinkfoot{the quasi-geostrophic equations in 
    Lab~\#8}{\LabeightURL}.

  An example of where the leap frog scheme is superior to the other
  first order schemes is for undamped periodic motion (which arose
  in the weather balloon example from
  Lab~\#1~\externalref{lab1:exm:balloon}).  This corresponds to the
  system of ordinary differential equations (with the damping
  parameter, $\beta$, taken to be zero):
  \[ \frac{dy}{dt} = u, \]
  \[ \frac{du}{dt} = - \frac{\gamma}{m} y. \]
  You've already discretized this problem using the forward difference
  formula, and the same can be done with the second order centered
  formula.   We can then compare the forward Euler and leap-frog
  schemes applied to this problem.  

  Solution plots are given in
  Figure~\ref{lab2:fig:test-leap-frog}, for parameters $\gamma/m=1$,
  $\Delta t=0.25$, $y(0)=0.0$ and $u(0)=1.0$, and demonstrate that the
  leap-frog scheme is stable, while forward Euler is unstable.
  This can easily be explained in terms of the stability criteria we
  derived for the two schemes when applied to the test equation.
  The undamped oscillator problem is a linear problem with pure
  imaginary eigenvalues, so as long as $|\sqrt{\gamma/m}\Delta t|<1$,
  the leap frog scheme is stable, which is obviously true for the
  parameter values we are given.  
  Furthermore, the forward Euler stability condition $|1+\lambda\Delta
  t|<1$ is violated for any choice of time step (when $\lambda$ is
  pure imaginary) and so this scheme is always unstable for the
  undamped oscillator.
  \begin{figure}[htbp]
    \begin{center}
      \leavevmode
      \htmlimage{scale=1.5}
%%%%%%%%%%%%%%%%%%%%%%%%%%%%%%%%%%%%%%%%%%%%%%%%%%%%%%%%%%%%%%%
%  NOTE: The following two plots are produced by output 
%  from the C program ``oscillator/osc'' (source ``osc.c'') 
%  which is run using the command ``osc 80 >o'' (80 time steps,
%  output file ``o'').  The postscript images are created 
%  using the gnuplot script ``gin''. 
%%%%%%%%%%%%%%%%%%%%%%%%%%%%%%%%%%%%%%%%%%%%%%%%%%%%%%%%%%%%%%%
      \includegraphics[height=3.0in]{oscillator/leap-frog}
      \includegraphics[height=3.0in]{oscillator/forward-euler}
      \caption{Numerical solution to the undamped harmonic oscillator
        problem, using the forward Euler and leap-frog schemes.
        Parameter values: $\gamma/m=1.0$, $\Delta t=0.25$, $y(0)=0$,
        $u(0)=1.0$.  The exact solution is a sinusoidal wave.}
      \label{lab2:fig:test-leap-frog}
    \end{center}
  \end{figure}

  \begin{note}
    Had we taken a larger time step (such as $\Delta t=2.0$, for
    example), then even the leap-frog scheme is unstable.
    Furthermore, if we add damping 
    ($\beta\neq 0$), then the eigenvalues are no longer pure imaginary,
    and the leap frog scheme is unstable no matter what time step we
    use.   
  \end{note}
\end{example}

%%%%%%%%%%%%%%%%%%%%%%%%%%%%%%%%%%%%%%%%%%%%%%%%%%%%%%%%%%%%%%%%%%%
\section{Stiff Equations}
\label{lab2:sec:stiffness}

One final note: this Lab has dealt only with ODE's (and systems of
ODE's) that are \emph{ non-stiff}.  \emph{ Stiff equations} are equations
that have solutions with at least two widely varying times scales over
which the solution 
changes.  An example of stiff solution behaviour is a problem with
solutions that have rapid, 
transitory oscillations, which die out over a short time scale, after
which the solution slowly decays to an equilibrium.  A small time step
is required in the initial transitory region in order to capture the
rapid oscillations.  However, a larger time step can be taken in the
non-oscillatory region where the solution is smoother.  Hence, using a
very small time step will result in very slow and inefficient
computations.  

There are also many other numerical schemes designed
specifically for stiff equations, most of which are implicit schemes.
We will not describe any of them 
here -- you can find more information in a numerical analysis text
such as ~\cite{burden-faires}.  

\begin{latexonly}
\gloss{stiffness}{when referring to a solution to a DE or a system of
  DE's, stiffness refers to a solution with at least two widely
  varying times scales over which the solution 
changes.}
\end{latexonly}

%%% Local Variables: 
%%% mode: latex
%%% TeX-master: "lab2"
%%% End: 
